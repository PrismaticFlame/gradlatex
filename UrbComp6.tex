\documentclass[12pt]{article}
\usepackage{styles/cwilliams-standard}

\setclass{\URBCOMP}
\settitle{Homework 6}

\begin{document}

\maketitlepage

\section{Mathematical Relationship}

Finding the relationship between the I-curve and the $R_0$ value requires taking the SIR model equation of the rate of change of I w.r.t time and adding the $R_0$ equation into the mix

\begin{align}
    \frac{dI}{dt} = \beta S I - \gamma I \quad
    & R_0 = \frac{\beta}{\gamma} \cdot N \\
    \frac{dI}{dt} = \beta S I - \gamma I \quad
    & \beta = \frac{R_0 \gamma}{N}
\end{align}

where $S$ is `susceptible', $I$ is infected, $\beta$ is the transmission rate, $\gamma$ is the recovery rate, and $N$ is the total population size.

After getting the $R_0$ equation in terms of $\beta$, we can substitute that into our SIR model equation and find the relationship.

\begin{align}
    \frac{dI}{dt} =& \bigl(\frac{R_0 \gamma}{N}\bigr) S I - \gamma I
\end{align}

\begin{equation}
    \boxed{\frac{dI}{dt} = \gamma I \bigl( \frac{R_0 S}{N} - 1 \bigr)}
\end{equation}

This equation reveals the relationship between the I curve and $R_0$. The term $\frac{R_0 S}{N}$ represents
the effective reproduction number, $R_{\text{eff}}$ or $R(t)$, which decreases over time as susceptible people become infected.

\textbf{Early in the epidemic ($S \approx N$)}: $R_{\text{eff}} \approx R_0 > 1$, so $\frac{dI}{dt} > 0$ and infections rise exponentially. 
The larger $R_0$ is, the faster this initial growth.

\textbf{At the peak ($S = N / R_0$)}: $R_{\text{eff}} = 1$, so $\frac{dI}{dt} = 0$ and infections stop increasing. Note that this occurs 
when only $\frac{1}{R_0}$ of the population remains susceptible.

\textbf{After the peak ($S < N/R_0$)}: $R_{\text{eff}} < 1$, so $\frac{dI}{dt} < 0$ and infections decline steeply as there aren't enough 
susceptibles left to sustain transmission.

\section{Influenza Outbreak}

\subsection{Compartmental SEIR Model Update}

To give insight into how the outbreak will unfold around schools, we will split the population into two: open and closed schools.
\begin{itemize}
    \item \textbf{Open schools} (subscript `O'): $S_o, E_o, I_o, R_o$
    \item \textbf{Closed schools} (subscript `C'): $S_c, E_c, I_c, R_c$
\end{itemize}

Then, from these new variables as part of the SEIR compartmental model, we can create new equations for these models in their respective populations.


\begin{figure}[H]
    \centering
    \begin{tabularx}{\textwidth}{@{}XX@{}}
        \subsection*{Open schools} &
        \subsection*{Closed schools} \\[1ex]
        %
        $\begin{aligned}
            \frac{dS_o}{dt} &= -\beta_{oo}S_o I_o - \beta_{oc} S_o I_c \\
            \frac{dE_o}{dt} &= \beta_{oo} S_o I_o + \beta_{oc} S_o I_c - \sigma E_o \\
            \frac{dI_o}{dt} &= \sigma E_o - \gamma I_o \\
            \frac{dR_o}{dt} &= \gamma I_o
        \end{aligned}$ &
        %
        $\begin{aligned}
            \frac{dS_c}{dt} &= -\beta_{co}S_c I_o - \beta_{cc} S_c I_c \\
            \frac{dE_c}{dt} &= \beta_{co} S_c I_o + \beta_{cc} S_c I_c - \sigma E_c \\
            \frac{dI_c}{dt} &= \sigma E_c - \gamma I_c \\
            \frac{dR_c}{dt} &= \gamma I_c
        \end{aligned}$
    \end{tabularx}
\end{figure}

These updated models will allow us to compare the resulting curves that come from school closures and continuing to keep schools open, 
and quantify the impact of school closures on reducing total infections. Some key things of note in this updated model is that I am accounting for
``community'' infections. Kids won't infect susceptibles at school, but they may continue to interact at parks, grocery stores, in sports, etc.

\begin{itemize}
    \item $\beta_{oo}$: Transmission rate among open-school students (school + community).
    \item $\beta_{cc}$: Transmission rate among closed-school students (community only).
    \item $\beta_{co}$: Transmission rate from closed-school to open-school students (community mixing).
    \item $\beta_{oc}$: Transmission rate from open-school to closed-school students (community mixing).
\end{itemize}

Something of note is that (I expect) $\beta_{oo} > \beta_{cc}$ because open school students would still be getting exposed
to other students, whereas the closed school students would be getting far less exposure in school. Also it's worth mentioning that $\beta_{co} \approx \beta{oc}$ 
as they are really measuring the same thing.

\subsection{Agent-based SEIR Model Update}

Keeping with the same idea as before, but applying it to individuals instead of entire `systems'. Each individual gets treated as an agent who has their own attributes:

\begin{itemize}
    \item \textbf{Category}: S, E, I, or R.
    \item \textbf{School}: Open, Closed.
    \item \textbf{Age group}: Student, Adult, Elderly.
    \item \textbf{Location}: Which school, household, neighborhood.
    \item \textbf{Contact Network}: List of other agents they interact with.
\end{itemize}

We would then construct a network of agents that all have this information stored with each agent. They also contain information on their 
contacts from each place they visit (household, school, and community).

This model would allow you to track individual-level outcomes and more accurately model the transmission of infection on an individual basis and with much 
greater information like who, (probability of) when, and from whom. You would also inadvertently create a network graph of who knows who and who spends time with who, creating 
heterogeneous contact patterns, and reveal clusterings (households, school cliques, community centers).

I remember that when COVID-19 was around, I was in my final year of High School. When schools eventually closed I spent a large amount of my time indoors and online (mainly based on preference). 
However, I had many friends that took the opportunity of `no more school' to spend time with friends outside of school at each others houses or outside in parks, increasing their social exposure. 
So, we could also reasonably see some behavioral changes where certain contacts are reduced (because interactions between agents don't happen at school anymore) and increased interactions 
with other agents (friends spending time together outside of school).


\begin{table}[H]
    \centering
    \begin{tabular}{cc}
        \toprule
        Compartmental Model & Agent-Based Model \\
        \midrule
        Quick insight into overall trends & Tracking of specific transmission chains \\
        Overall trend tracking & Targeted interventions \\
        \bottomrule
    \end{tabular}
\end{table}

\section{Epidemiological Models}

\begin{figure}[H]
    \centering
    \begin{tabularx}{\textwidth}{@{}XX@{}}
        \subsection{Individuals Can Recover?}
        \textbf{Models: SIR, SIRS, SEIR}
        
        Models that include \textbf{R} (Recovery) indicate that individuals can recover from infection, though not necessarily with permanent immunity.
        &
        \subsection{Monotonically Decreasing Susceptibility?}
        \textbf{Models: SI, SIR, SEIR}
        
        These models feature monotonically decreasing susceptibility because individuals do not return to the \textbf{S} category. Once removed from the susceptible population, they remain permanently infected or recovered.
        \\[2em]
        
        \subsection{Permanent Immunity?}
        \textbf{Models: SIR, SEIR}
        
        Models that end with Recovery indicate that once an individual recovers, they acquire permanent immunity and will not become infected again.
        &
        \subsection{Equilibrium?}
        \textbf{Models: SIS, SIRS}
        
        Equilibrium between \textbf{S} and \textbf{I} is only possible when individuals can return to the susceptible category. These models allow such transitions, enabling a stable endemic state.
        \\[2em]
        
        \subsection{Repeated Infections?}
        \textbf{Models: SIS, SIRS}
        
        Models that end with \textbf{S} (Susceptible) indicate that individuals will either become infected again or lose their immunity after recovery, returning to the susceptible state.
        &
        \subsection{Natural Decline of Infection?}
        \textbf{Models: SIR, SEIR, SI}
        
        These models feature natural infection decline. In SIR and SEIR, individuals recover and the infection eventually ends/dies. The SI model is debatable: while everyone becomes infected, one could argue that if everyone is infected, no one is infected.
        \\
    \end{tabularx}
\end{figure}

\section{Ebola and COVID-19 Paper Review}

\subsection{\textit{Epidemic of Ebola}}

\subsection{\textit{COVID-19 Policy Response}}

\end{document}