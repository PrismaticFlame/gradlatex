\documentclass[12pt]{article}
\usepackage{styles/cwilliams-standard}

\setclass{\URBCOMP}
\settitle{Homework 6}

\begin{document}

\maketitlepage

\section{Mathematical Relationship}

\section{Influenza Outbreak}

\subsection{Compartmental SEIR Model Update}

\subsection{Agent-based SEIR Model Update}

\section{Epidemiological Models}

\begin{figure}[H]
    \begin{minipage}{.48\textwidth}
        \subsection{Individuals can recover?}

        \textbf{SIR, SIRS, SEIR.} These models that include \textbf{R}, or Recovery, in them indicate that Individuals can recover. Not necessarily forever, but they can recover.

        \subsection{Permanent Immunity?}

        \textbf{SIR, SEIR}. These models that end with Recovery indicate that once an indiviudal recovers, they will not be sick again and will have permanent immunity to that infection.

        \subsection{Repeated Infections?}

        \textbf{SIS, SIRS}. These models that end with \textbf{S}, or Susceptible, indicate that individuals will either become infected again or lose their immunity to the infection after recovery.

    \end{minipage} %
    \begin{minipage}{.48\textwidth}
        \subsection{Monotonically Decreasing Susceptibility?}

        \textbf{SI, SIR, SEIR}. What determines the monotonically decreasing susceptible category is if inidividuals can return to the susceptible category or not. 
        These three models end do have indicate that individuals will return to the \textbf{S} category again, meaning that S will only ever decrease over time.

        \subsection{Equilibrium?}

        \textbf{SIS, SIRS}. Along the same line of thinking as in the previous question, the only way there can be an  equilibrium between
        \textbf{S} and \textbf{I} is if individuals can return to the susceptible category. Since individuals do return to susceptible with these models,
        there can be an equilibrium between the \textbf{S} and \textbf{I} category.

        \subsection{Natural Decline of Infection?}

        \textbf{SIR, SEIR, SI}. These models conclude with individuals recovering, meaning that (baring new additions or new births), the S category will never
        increase and therefore the infection will end at some point. However, the \textbf{SI} model could be included because the
        model ends with everyone being infected, which then comes to a point of interpretation: If everyone is infected, is \textit{anyone} infected?
    \end{minipage}
\end{figure}

\section{Ebola and COVID-19 Paper Review}

\subsection{\textit{Epidemic of Ebola}}

\subsection{\textit{COVID-19 Policy Response}}

\end{document}