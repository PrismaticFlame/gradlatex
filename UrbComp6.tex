\documentclass[12pt]{article}
\usepackage{styles/cwilliams-standard}

\setclass{\URBCOMP}
\settitle{Homework 6}

\begin{document}

\maketitlepage

\section{Mathematical Relationship}

Finding the relationship between the I-curve and the $R_0$ value requires taking the SIR model equation of the rate of change of I w.r.t time and adding the $R_0$ equation into the mix

\begin{align}
    \frac{dI}{dt} = \beta S I - \gamma I \quad
    & R_0 = \frac{\beta}{\gamma} \cdot N \\
    \frac{dI}{dt} = \beta S I - \gamma I \quad
    & \beta = \frac{R_0 \gamma}{N}
\end{align}

where $S$ is `susceptible', $I$ is infected, $\beta$ is the transmission rate, $\gamma$ is the recovery rate, and $N$ is the total population size.

After getting the $R_0$ equation in terms of $\beta$, we can substitute that into our SIR model equation and find the relationship.

\begin{align}
    \frac{dI}{dt} =& \bigl(\frac{R_0 \gamma}{N}\bigr) S I - \gamma I
\end{align}

\begin{equation}
    \boxed{\frac{dI}{dt} = \gamma I \bigl( \frac{R_0 S}{N} - 1 \bigr)}
\end{equation}

This equation reveals the relationship between the I curve and $R_0$. The term $\frac{R_0 S}{N}$ represents
the effective reproduction number, $R_{\text{eff}}$ or $R(t)$, which decreases over time as susceptible people become infected.

\textbf{Early in the epidemic ($S \approx N$)}: $R_{\text{eff}} \approx R_0 > 1$, so $\frac{dI}{dt} > 0$ and infections rise exponentially. 
The larger $R_0$ is, the faster this initial growth.

\textbf{At the peak ($S = N / R_0$)}: $R_{\text{eff}} = 1$, so $\frac{dI}{dt} = 0$ and infections stop increasing. Note that this occurs 
when only $\frac{1}{R_0}$ of the population remains susceptible.

\textbf{After the peak ($S < N/R_0$)}: $R_{\text{eff}} < 1$, so $\frac{dI}{dt} < 0$ and infections decline steeply as there aren't enough 
susceptibles left to sustain transmission.

\section{Influenza Outbreak}

\subsection{Compartmental SEIR Model Update}



\subsection{Agent-based SEIR Model Update}

\section{Epidemiological Models}

\begin{figure}[H]
    \centering
    \begin{tabularx}{\textwidth}{@{}XX@{}}
        \subsection{Individuals Can Recover?}
        \textbf{Models: SIR, SIRS, SEIR}
        
        Models that include \textbf{R} (Recovery) indicate that individuals can recover from infection, though not necessarily with permanent immunity.
        &
        \subsection{Monotonically Decreasing Susceptibility?}
        \textbf{Models: SI, SIR, SEIR}
        
        These models feature monotonically decreasing susceptibility because individuals do not return to the \textbf{S} category. Once removed from the susceptible population, they remain permanently infected or recovered.
        \\[2em]
        
        \subsection{Permanent Immunity?}
        \textbf{Models: SIR, SEIR}
        
        Models that end with Recovery indicate that once an individual recovers, they acquire permanent immunity and will not become infected again.
        &
        \subsection{Equilibrium?}
        \textbf{Models: SIS, SIRS}
        
        Equilibrium between \textbf{S} and \textbf{I} is only possible when individuals can return to the susceptible category. These models allow such transitions, enabling a stable endemic state.
        \\[2em]
        
        \subsection{Repeated Infections?}
        \textbf{Models: SIS, SIRS}
        
        Models that end with \textbf{S} (Susceptible) indicate that individuals will either become infected again or lose their immunity after recovery, returning to the susceptible state.
        &
        \subsection{Natural Decline of Infection?}
        \textbf{Models: SIR, SEIR, SI}
        
        These models feature natural infection decline. In SIR and SEIR, individuals recover and the infection eventually ends/dies. The SI model is debatable: while everyone becomes infected, one could argue that if everyone is infected, no one is infected.
        \\
    \end{tabularx}
\end{figure}

\section{Ebola and COVID-19 Paper Review}

\subsection{\textit{Epidemic of Ebola}}

\subsection{\textit{COVID-19 Policy Response}}

\end{document}