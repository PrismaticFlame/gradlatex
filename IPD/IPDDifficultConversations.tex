\documentclass[12pt]{article}
\usepackage{styles/cwilliams-standard}
% \usepackage{enumitem}
% \usepackage{parskip}

\setclass{\IPD}
\settitle{Difficult Conversations Workshop}

\begin{document}

\maketitlepage

\section{First Situation: Manager Perspective}

\subsection{Identify the type of challenge(s) occurring in this situation}

\textbf{What:} A technical lead verbally attacked junior team members during a stand-up meeting, using inappropriate language and disparaging their work in front of the entire team.

\textbf{How:} The challenge manifested as an emotional outburst that violated team norms around respect, psychological safety, and professionalism. The behavior created an uncomfortable silence and damaged team dynamics.

\textbf{Why:} The underlying challenges likely include:
\begin{itemize}[nosep]
    \item High stress and pressure from the high-visibility project
    \item Long working hours leading to potential burnout
    \item Anxiety and frustration with project progress
    \item Possible concerns about work quality or team capability
    \item Communication breakdown about workload or expectations
\end{itemize}

\subsection{Questions to ask the team member about his behavior}

Questions to ask the technical lead:
\begin{itemize}[nosep]
    \item ``Can you help me understand what was going on for you in that moment?''
    \item ``What's been weighing on you regarding this project?''
    \item ``How are you feeling about your workload and the project timeline?''
    \item ``What support do you need from me as your manager?''
    \item ``Do you understand how your words impacted the team members?''
    \item ``What was your intention in that moment, and do you think it achieved that goal?''
    \item ``Are you aware of how this violated our team norms around respect and psychological safety?''
\end{itemize}

\subsection{How and when to bring up these questions}

\textbf{When:} I would address this in two stages:
\begin{enumerate}[nosep]
    \item Immediately after the meeting (within 30 minutes), I would briefly pull the technical lead aside for a quick private conversation to pause the behavior and check in.
    \item Schedule a formal one-on-one meeting later that same day or the next morning for a deeper conversation.
\end{enumerate}

\textbf{How:}
\begin{itemize}[nosep]
    \item \textit{Start by stating the facts:} ``In today's stand-up, you raised your voice, used inappropriate language, and criticized junior team members' work publicly.''
    \item \textit{Share my concern:} ``I'm concerned because this violated our team norms around respect and psychological safety, and I could see the impact it had on the team.''
    \item \textit{Ask questions to understand:} Use the questions listed above to understand the root causes.
    \item \textit{Work toward a solution:} Discuss expectations going forward, support needed, and steps for repair (including an apology to the team).
    \item \textit{Follow up:} Address burnout indicators and workload concerns.
\end{itemize}

\subsection{Team norms to create for accountability}

Team norms to establish:
\begin{itemize}[nosep]
    \item \textbf{Respect:} We treat all team members with respect, regardless of experience level. We critique work, not people.
    \item \textbf{Psychological Safety:} Team members should feel safe to share ideas, ask questions, and make mistakes without fear of humiliation.
    \item \textbf{Constructive Feedback:} Feedback should be specific, actionable, and delivered privately when it involves criticism of individual work.
    \item \textbf{Inclusive Communication:} We encourage authenticity while maintaining professionalism. We value all voices and perspectives.
    \item \textbf{Accountability:} When team norms are violated, we address them promptly and directly. Team members can call out behavior that doesn't align with our values.
    \item \textbf{Support System:} Team members experiencing stress or burnout should feel comfortable bringing it up early so we can address it together.
    \item \textbf{Public Accountability Process:} Using the ``point it out, check it out, work it out'' approach when norms are violated in team settings.
\end{itemize}

\section{Second Situation: Junior Team Member Perspective}

\subsection{Identify the type of challenge(s) occurring in this situation}

\textbf{What:} The lead software engineer publicly criticized and disparaged you and other team members using inappropriate language during a stand-up meeting.

\textbf{How:} The challenge involves:
\begin{itemize}[nosep]
    \item Violation of team norms around respect and professionalism
    \item Public humiliation and inappropriate criticism
    \item Lack of support and clear direction prior to the outburst
    \item Creating a hostile and psychologically unsafe work environment
\end{itemize}

\textbf{Why:} The underlying issues include:
\begin{itemize}[nosep]
    \item Breakdown in communication and support systems
    \item Possible bias or neglect toward junior/new team members
    \item Lead engineer's stress and burnout manifesting as inappropriate behavior
    \item Management's failure to ensure proper onboarding and support
    \item Gap between expected team norms (respect, inclusion, support) and actual behavior
\end{itemize}

\subsection{Addressing the gap with the manager}

I would request a private one-on-one meeting with the manager and use the following approach:

\textbf{State the facts:}

``I'd like to talk to you about something that happened in yesterday's stand-up meeting. The lead software engineer raised his voice, used inappropriate language, and publicly criticized me and other junior team members, saying our work was [bad word] and that we don't know what we're doing.''

\textbf{Describe the pattern/gap:}

``I've also noticed a pattern over the past few weeks where the lead engineer has been less willing to help me, and I haven't been receiving clear direction on what to do. I'm concerned because this doesn't align with the team values of respect, psychological safety, and support that I expected when I joined.''

\textbf{Share my story/concern:}

``I'm concerned that this behavior is creating an environment where junior team members don't feel valued or safe to ask questions and learn. It's also making it difficult for me to contribute effectively to this important project. I'm worried that without support and guidance, I won't be able to do my best work, and it sends the message that new team members aren't valued.''

\textbf{Invite dialogue:}

``Can you help me understand how you see this situation? What are your expectations for how team members should support each other, especially when it comes to helping newer members get up to speed?''

\textbf{Seek resolution:}

``What can we do to address this? I want to contribute meaningfully to the team, but I need clearer direction and a more supportive environment. Can we also discuss what the appropriate team norms should be and how we can hold everyone accountable to them?''

\section{Third Situation: Lead Software Engineer Perspective}

\subsection{Identify the type of challenge(s) occurring in this situation}

\textbf{What:} I had an emotional outburst during a stand-up meeting where I criticized team members' capabilities and expressed frustration about being the only competent person on the team.

\textbf{How:} The challenge manifested as:
\begin{itemize}[nosep]
    \item Loss of emotional control in a public setting
    \item Inappropriate expression of frustration
    \item Violation of team norms around respect and professionalism
    \item Damage to team relationships and psychological safety
    \item Public criticism of colleagues' competence
\end{itemize}

\textbf{Why:} The underlying issues include:
\begin{itemize}[nosep]
    \item \textbf{Burnout:} Long hours, nights, and overwhelming responsibility for project outcomes
    \item \textbf{Workload stress:} Heavy existing workload with additional tasks being added
    \item \textbf{Perceived burden:} Feeling responsible not only for my own work but for the quality of others' work
    \item \textbf{Training frustration:} Time spent training team members on what I consider basic skills
    \item \textbf{Lack of support:} Feeling alone in carrying the project's success
    \item \textbf{Communication breakdown:} Not expressing concerns about workload and capacity to my manager before reaching a breaking point
    \item \textbf{Unrealistic expectations:} Possibly holding team members to unrealistic standards or not understanding their actual skill levels and learning curves
\end{itemize}

\subsection{Questions to ask yourself about your own behavior}

Self-reflection questions:
\begin{itemize}[nosep]
    \item ``What triggered my emotional response in that specific moment?''
    \item ``Was I fair in my assessment of my team members' capabilities?''
    \item ``Have I clearly communicated my concerns about workload and support to my manager before this outburst?''
    \item ``Am I experiencing burnout, and what are the signs I've been ignoring?''
    \item ``What would have been a more constructive way to express my frustrations?''
    \item ``How did my words impact my team members, and what damage have I done to team trust and psychological safety?''
    \item ``Are my expectations of my team members realistic given their experience levels and the training they've received?''
    \item ``What support do I actually need from my manager to succeed in this role?''
    \item ``Am I taking on too much responsibility by trying to control every aspect of the project?''
    \item ``How can I repair the relationships I've damaged with my team members?''
    \item ``What changes do I need to make to prevent this from happening again?''
\end{itemize}

\subsection{How and when to talk to your manager about burnout}

\textbf{When:}
\begin{itemize}[nosep]
    \item I would request a one-on-one meeting with my manager as soon as possible, ideally within the next day or two.
    \item I would also first apologize to the team for my outburst, acknowledging that my behavior was inappropriate and violated team norms, before having the conversation with my manager.
\end{itemize}

\textbf{How:}

I would approach this conversation with the mindset of being ``a contributor craving focus, not a complainer craving less work.''

\textbf{Opening:}

``Thank you for meeting with me. I want to talk to you about what happened in yesterday's stand-up meeting and some challenges I've been facing. First, I want to acknowledge that my behavior was completely inappropriate. I let my frustration get the best of me, and I said things that were disrespectful to my team members. I've already apologized to the team, but I wanted you to hear directly from me that I take full responsibility for that outburst.''

\textbf{State the facts about workload/burnout:}

``I've been working long days and nights trying to deliver on this high-visibility project. On top of my existing technical responsibilities, I've been given additional tasks, and I've been spending significant time training team members. In yesterday's meeting when I was assigned even more tasks, I reached my breaking point.''

\textbf{Share my story/concern (as a contributor):}

``I'm concerned because I want to deliver excellent results on this project, but I'm stretched too thin to maintain the quality and focus needed. I'm trying to be responsible for every aspect of the project outcome, and I realize that's not sustainable. I want to focus my energy on the areas where I can add the most value to what you and the organization care about most.''

\textbf{Invite dialogue:}

``Can we talk about priorities and focus? What are the most critical things you need from me on this project? Where can I be most valuable?''

\textbf{Seek solutions:}

``I'd like to discuss:
\begin{itemize}[nosep]
    \item Which tasks are truly my highest priorities so I can focus my energy there
    \item What support or resources might be available to help with the workload
    \item How we can better distribute responsibilities across the team
    \item What training or support the junior team members might need that doesn't have to come solely from me
    \item What reasonable boundaries I can set around my working hours to prevent burnout
\end{itemize}

I'm committed to this project's success, but I need to find a more sustainable way to contribute effectively.''

\textbf{Follow-up commitment:}

``I also want to commit to communicating earlier when I'm feeling overwhelmed in the future, rather than letting it build up to a breaking point. Can we establish regular check-ins to discuss workload and any concerns before they become critical?''

\end{document}