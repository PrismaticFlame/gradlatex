\documentclass[12pt]{article}
\usepackage{styles/cwilliams-standard}

% \usepackage[margin=1in]{geometry}

\setclass{\URBCOMP}
\settitle{Economic-Based Population Flow Indicator \\ \vspace{0.3cm}
 \large{Economic Strength as an Indicator of Population Flow}}


\begin{document}

\maketitlepage

\section{Abstract}

\section{Relevant Work}

\section{Necessary Data (Preliminary)}

Data that would be necessary for prediction has a range of viability. This ranges from absolutely 
necessary to ``could be nice'' but nigh impossible to quantify.

\begin{table}[H]
    \centering
    \begin{tabular}{c|ccc}
        \toprule
        \textbf{Data} & \textbf{Necessity} & \textbf{Quantification} & \textbf{Viability} \\
        \midrule
        
    \end{tabular}
\end{table}

\section{Datasets (Preliminary)}

\begin{description}
    \item[IMF World Economic Outlook (WEO):] ``The World Economic Outlook
            (WEO) database is created during the biannual WEO exercise, which begins in January
            and June of each year and results in the April
            and September/October WEO publication. Selected series from the publication are available
            in a database format''. \href{https://data.imf.org/en/datasets/IMF.RES:WEO}{WEO Dataset}
    \item[UN Tourism Statistics Database:] ``UN
            Tourism systematically collects tourism statistics from countries and territories around the
            world in an extensive database that provides the
            most comprehensive repository of statistical information available on the tourism sector''.
            \href{https://www.untourism.int/tourism-statistics/tourism-statistics-database}{UN Tourism Database}
\end{description}

\section{Potential Needs}

There is the possibility of needing to \textit{create} a database that has the exchange rates of currencies
between all countries (top 100 countries by GDP). This would allow for greater expansion of seeing which countries are doing well
economically and which countries are not comparatively, and then determine which demographics of people are going where. E.g., China has 
prospered and grown economically in the past 50 years, and retired, elderly Chinese people are being encouraged to tour the world. The CNY (¥) has strengthened, so large groups
of elderly Chinese are exploring Europe and the U.S. Conversely, the Japanese Yen (¥) has weakened in the past 20 years and 
Americans are either moving to or touring Japan.

Having an expansive database that is able to catalogue exchange rates between countries that is not dollar (\$) centric 
would allow for drawing broader conclusions across the world.

\end{document}