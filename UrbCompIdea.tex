\documentclass[12pt]{article}
\usepackage{styles/cwilliams-standard}

% \usepackage[margin=1in]{geometry}

\setclass{\URBCOMP}
\settitle{Economic-Based Population Flow Indicator \\ \vspace{0.3cm}
 \large{Economic Strength as an Indicator of Population Flow}}


\begin{document}

\maketitlepage

\section{Abstract}\label{sec:abstract}

In current study of mobility flow, there is a common trend of utilizing large indicators that do not accurately
show base trends. Projects that have used GDP, average/median household income, population density, and land use are using
data that is inherently a black box and is regularly abused by countries and superpowers. 

When tariffs where imposed by the 47th President of the United States, companies within the U.S., in panic, stocked up and imports went up drastically. 
While this is not a good thing, on paper and in the GDP calculation it appeared that the U.S. was doing better than ever because GDP rose. Short term gain for
long term pain. This kind of data that can obfuscate the real factors that determine the movements of peoples are
difficult to utilize effectively.

This project proposes using currency exchange rates as a way to predict and analyze international population flows. 
When a currency strengthens, people from that country travel more—like how elderly Chinese tourists have been exploring Europe 
and the US as the Yuan has grown stronger. When a currency weakens, it attracts inbound visitors—similar to how Americans are 
increasingly visiting Japan as the Yen has declined.

Using machine learning, we could reasonably create a model that will predict broad population movements based on 
economic trends in countries around the world (much like \textit{Psychohistory} from \textit{Foundation} by Isaac Asimov!).
Utilizing databases from the IMF, UN, and other reputable sources (listed in Section \ref{sec:data}) will be used to compare
economic strengths of countries and be used to determine population flow.

\section{Relevant Work (Preliminary)}\label{sec:relwork}

As stated above, there have been projects and studies that utilize factors that inherently have a difficult time capturing underlying trends and issues.
However, it is necessary to see thet techniques employed that got the results they achieved. As well as to understand where the current
literature is at and not reinvent the wheel where possible. Here is a list of papers that are related to the subject:

\begin{description}
    \item[A Machine Learning Approach to Modeling Human Migration \cite{robinson2017mlhumanmigration}:] Using socioeconomic factors and environmental characteristics like land use, GDP, and median income,
    and utilizing machine learning, it was possible to predict inter-county and international migration flows with high accuracy and can be used under `what-if' scenarios
    like `what if sea levels rise'.
    \item[Forecasting asylum-related migration flows with machine learning and data at scale \cite{carammia2022forecastingasylum}:] The migration crisis of 2015-2016 in the European Union (EU) left admin's. unprepared, 
    exposing shortcomings in migration forecasting. This study proposes a machine learning model that takes in administrative statistics as well as 
    non-traditional data sources at scale to accurately prediction asylum seeking applications in the EU but can be expanded to other countries and any
    context given enough information.
    \item[Revisiting forced migration: A machine learning perspective \cite{micevska2021forcedmigration}:] Conflict is a major cause of migration and movment between countries. If 
    there is conflict, it will force people away, or to migrate. Also, this project uses is the first to use internet penetration rate in explain migration flows.
\end{description}

\section{Necessary Data (Preliminary)}\label{sec:necdata}

These are potential options for data that would be used in indicating the flow of population with economic indicators. 

\begin{itemize}
    \item \textbf{Necessity}: How necessary the data is to the prediction. Ranging from `Necessary' to `Unlikely Necessary'.
    \item \textbf{Quantification}: How quantifiable the data is. Ranging from `Quantifiable' to `Difficult to Quantify'.
    \item \textbf{Viability}: How viable the data would be to use in prediction. Ranging from `Viable' to `Unlikely Viable'.
\end{itemize}

\begin{table}[H]
    \centering
    \begin{tabular}{c@{\hspace{0.5em}}c|ccc}
        \toprule
        & \textbf{Data} & \textbf{Necessity} & \textbf{Quantification} & \textbf{Viability} \\
        \midrule
        \multirow{4}{*}{\rotatebox[origin=c]{90}{Best}} 
        & \textcolor{green!60!black}{Exchange Rates} & Necessary & Quantifiable & Viable \\
        & \textcolor{green!60!black}{Tourism/Immigration} & Necessary & Quantifiable & Viable \\
        & \textcolor{green!60!black}{War/Conflict} & Necessary & Quantifiable & Viable \\
        & \textcolor{green!60!black}{Tariffs} & Necessary & Somewhat Quantifiable & Viable \\
        \cmidrule{2-5}
        \multirow{6}{*}{\rotatebox[origin=c]{90}{Decent}} 
        & \textcolor{blue!60!black}{Visa Policy} & Potentially Necessary & Quantifiable & Viable \\
        & \textcolor{blue!60!black}{Presence of Int'l Chains} & Potentially Necessary & Quantifiable & Viable \\
        & \textcolor{blue!60!black}{Standard Goods Prices} & Somewhat Necessary & Quantifiable & Potenitally Viable \\
        & \textcolor{blue!60!black}{Job Opportunity} & Potentially Necessary & Quantifiable & Potentially Viable \\
        & \textcolor{blue!60!black}{Age of Inflow/Outflow} & Potentially Necessary & Quantifiable & Potentially Viable \\
        & \textcolor{blue!60!black}{Int'l Alliance Proximity} & Potentially Necessary & Quantifiable & Potentially Viable \\
        \cmidrule{2-5}
        \multirow{4}{*}{\rotatebox[origin=c]{90}{Okay}} 
        & \textcolor{orange!60!black}{Country-Country Ties} & Potentially Necessary & Somewhat Quantifiable & Potentially Viable \\
        & \textcolor{orange!60!black}{Attractions} & Potentially Necessary & Somewhat Quantifiable & Potentially Viable \\
        & \textcolor{orange!60!black}{Safety} & Potentially Necessary & Difficult to Quantify & Potentially Viable \\
        & \textcolor{orange!60!black}{Internal Stability} & Potentially Necessary & Difficult to Quantify & Unlikely Viable \\
        \cmidrule{2-5}
        \multirow{2}{*}{\rotatebox[origin=c]{90}{Weak}} 
        & \textcolor{red!60!black}{Cultural Exports} & Unlikely Necessary & Difficult to Quantify & Unlikely Viable \\
        & \textcolor{red!60!black}{Demographics of Inflow} & Unlikely Necessary & Difficult to Quantify & Unlikely Viable \\
        \bottomrule
    \end{tabular}
\end{table}

\section{Datasets (Preliminary)}\label{sec:data}

The datasets below are only pretaining to the potential data needs in the `Best' category. If it is determined that other
datasets are required from other categories then they will be added in the future.

\begin{description}
    \item[IMF World Economic Outlook (WEO):] ``The World Economic Outlook
            (WEO) database is created during the biannual WEO exercise, which begins in January
            and June of each year and results in the April
            and September/October WEO publication. Selected series from the publication are available
            in a database format''. \href{https://data.imf.org/en/datasets/IMF.RES:WEO}{WEO Dataset}
    \item[UN Tourism Statistics Database:] ``UN
            Tourism systematically collects tourism statistics from countries and territories around the
            world in an extensive database that provides the
            most comprehensive repository of statistical information available on the tourism sector''.
            \href{https://www.untourism.int/tourism-statistics/tourism-statistics-database}{UN Tourism Database}
    \item[Our World in Data - Tourism:] ``Tourism has massively increased in recent decades. Aviation has opened up travel from domestic to international. Before the COVID-19 pandemic, 
    the number of international visits had more than doubled since 2000. ... On this page, you can find data and visualizations on the history and current state of tourism across the world.''
    \href{https://ourworldindata.org/tourism#all-charts}{Our World in Data - Tourism}
    \item[World Trade Organization:] ``The Tariff \& Trade Data platform provides official tariff and import data for more than 150 economies, including
    annual data from 1996 onwards for many of them. ...'' \href{https://ttd.wto.org/en}{World Trade Organization - Trade \& Tariff}
    \item[The Correlates of War Project:] This project collects and disseminates reliable and quantitative data 
            in international relations the include Trade, Arms Technology, War Data (1816-2007), Militarized Interstate Disputes, and others.
            This project can be used to find conflicts that are happening around the world as well as what spheres influence and capabilites countries have.
            \href{https://correlatesofwar.org/data-sets/}{Correlates of War}
    \item[ACLED:] ``ACLED is an independent, impartial conflict monitor providing real-time data and analysis on violent conflict and protest in all countries and territories across the world.'' \href{https://acleddata.com/}{ACLED Website}
\end{description}

\section{Potential Needs}\label{sec:needs}

There is the possibility of needing to \textit{create} a database that has the exchange rates of currencies
between all countries (top 100 countries by GDP). This would allow for greater expansion of seeing which countries are doing well
economically and which countries are not comparatively, and then determine which demographics of people are going where. E.g., China has 
prospered and grown economically in the past 50 years, and retired, elderly Chinese people are being encouraged to tour the world. The CNY (¥) has strengthened, so large groups
of elderly Chinese are exploring Europe and the U.S. Conversely, the Japanese Yen (¥) has weakened in the past 20 years and 
Americans are either moving to or touring Japan.

Having an expansive database that is able to catalogue exchange rates between countries that is not dollar (\$) centric 
would allow for drawing broader conclusions across the world.

Technically, this would be done within something akin to how Google Translate works: To get better translations (exchange rates) without
exponentially increasing space complexity, there needs to be a `third' language (currency) in the mixed. If you wanted to translate from 
Madarin to English (Yuan to USD), there needs to be a third language that only the model database/model would know that can do the 
communication for you. Instead of storing every exchange rate between every country and updating that consistently, there would be a third
currency in each equation that would be say `1 Yuan is worth X amount of inter-currency currency (ICC), and 1 Dollar is worth X amount of ICC,
therefore the exchange rate is Y'.

\bibliography{sources}

\end{document}