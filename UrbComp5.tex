\documentclass[12pt]{article}
\usepackage{cwilliams-standard}

\setclass{\URBCOMP}
\settitle{Homework 5}

\begin{document}

\maketitlepage

\section{3x4 Real Matrix}

To construct a 3x4 real, non-zero matrix, we must start with two vectors that are both real and non-zero.
Then, we multiply these vectors together to get our $A$ matrix.

\begin{align}
    u = \begin{bmatrix} 1 \\ 2 \\ 3 \end{bmatrix} \quad
    v = \begin{bmatrix} 1 \\ 1 \\ 1 \\ 1 \end{bmatrix} \\
    A = u \cdot v^T = \begin{bmatrix}
        1 & 1 & 1 & 1 \\
        2 & 2 & 2 & 2 \\
        3 & 3 & 3 & 3 \\
    \end{bmatrix}
\end{align}

Now that we have our A matrix, we must remember the SVD equation: $A = U \Sigma V^T$. To get one
singular value in the SVD reconstruction that yields A exactly, we should start with finding $V^T$, but
more specifically $V$ which is defined as $V = A^T A$.

\begin{align}
    V = A^T A = \begin{bmatrix}
        1 & 2 & 3 \\
        1 & 2 & 3 \\
        1 & 2 & 3 \\
        1 & 2 & 3
    \end{bmatrix}
    \begin{bmatrix}
        1 & 1 & 1 & 1 \\
        2 & 2 & 2 & 2 \\
        3 & 3 & 3 & 3 \\
    \end{bmatrix}
    =
    \begin{bmatrix}
        14 & 14 & 14 & 14 \\
        14 & 14 & 14 & 14 \\
        14 & 14 & 14 & 14 \\
        14 & 14 & 14 & 14 \\
    \end{bmatrix}
\end{align}

What an easy matrix to deal with! Given that V is defined as "an $n \times n$ orthogonal matrix
whose columns are unit eigenvectors of $A^T A$, and that this is a rank 1 matrix 

\section{Clustering of 1D Dataset}

\section{Urban Mobility Matrix}

\section{Citibike Dataset (again...)}

\subsection{K-Mean Clustering}

\subsection{K-Means++}

\subsection{}

\end{document}