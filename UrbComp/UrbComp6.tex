\documentclass[12pt]{article}
\usepackage{styles/cwilliams-standard}

\setclass{\URBCOMP}
\settitle{Homework 6}

\begin{document}

\maketitlepage

\section{Mathematical Relationship}

Finding the relationship between the I-curve and the $R_0$ value requires taking the SIR model equation of the rate of change of I w.r.t time and adding the $R_0$ equation into the mix

\begin{align}
    \frac{dI}{dt} = \beta S I - \gamma I \quad
    & R_0 = \frac{\beta}{\gamma} \cdot N \\
    \frac{dI}{dt} = \beta S I - \gamma I \quad
    & \beta = \frac{R_0 \gamma}{N}
\end{align}

where $S$ is `susceptible', $I$ is infected, $\beta$ is the transmission rate, $\gamma$ is the recovery rate, and $N$ is the total population size.

After getting the $R_0$ equation in terms of $\beta$, we can substitute that into our SIR model equation and find the relationship.

\begin{align}
    \frac{dI}{dt} =& \bigl(\frac{R_0 \gamma}{N}\bigr) S I - \gamma I
\end{align}

\begin{equation}
    \boxed{\frac{dI}{dt} = \gamma I \bigl( \frac{R_0 S}{N} - 1 \bigr)}
\end{equation}

This equation reveals the relationship between the I curve and $R_0$. The term $\frac{R_0 S}{N}$ represents
the effective reproduction number, $R_{\text{eff}}$ or $R(t)$, which decreases over time as susceptible people become infected.

\textbf{Early in the epidemic ($S \approx N$)}: $R_{\text{eff}} \approx R_0 > 1$, so $\frac{dI}{dt} > 0$ and infections rise exponentially. 
The larger $R_0$ is, the faster this initial growth.

\textbf{At the peak ($S = N / R_0$)}: $R_{\text{eff}} = 1$, so $\frac{dI}{dt} = 0$ and infections stop increasing. Note that this occurs 
when only $\frac{1}{R_0}$ of the population remains susceptible.

\textbf{After the peak ($S < N/R_0$)}: $R_{\text{eff}} < 1$, so $\frac{dI}{dt} < 0$ and infections decline steeply as there aren't enough 
susceptibles left to sustain transmission.

\section{Influenza Outbreak}

\subsection{Compartmental SEIR Model Update}

To give insight into how the outbreak will unfold around schools, we will split the population into two: open and closed schools.
\begin{itemize}
    \item \textbf{Open schools} (subscript `O'): $S_o, E_o, I_o, R_o$
    \item \textbf{Closed schools} (subscript `C'): $S_c, E_c, I_c, R_c$
\end{itemize}

Then, from these new variables as part of the SEIR compartmental model, we can create new equations for these models in their respective populations.


\begin{figure}[H]
    \centering
    \begin{tabularx}{\textwidth}{@{}XX@{}}
        \subsection*{Open schools} &
        \subsection*{Closed schools} \\[1ex]
        %
        $\begin{aligned}
            \frac{dS_o}{dt} &= -\beta_{oo}S_o I_o - \beta_{oc} S_o I_c \\
            \frac{dE_o}{dt} &= \beta_{oo} S_o I_o + \beta_{oc} S_o I_c - \sigma E_o \\
            \frac{dI_o}{dt} &= \sigma E_o - \gamma I_o \\
            \frac{dR_o}{dt} &= \gamma I_o
        \end{aligned}$ &
        %
        $\begin{aligned}
            \frac{dS_c}{dt} &= -\beta_{co}S_c I_o - \beta_{cc} S_c I_c \\
            \frac{dE_c}{dt} &= \beta_{co} S_c I_o + \beta_{cc} S_c I_c - \sigma E_c \\
            \frac{dI_c}{dt} &= \sigma E_c - \gamma I_c \\
            \frac{dR_c}{dt} &= \gamma I_c
        \end{aligned}$
    \end{tabularx}
\end{figure}

These updated models will allow us to compare the resulting curves that come from school closures and continuing to keep schools open, 
and quantify the impact of school closures on reducing total infections. Some key things of note in this updated model is that I am accounting for
``community'' infections. Kids won't infect susceptibles at school, but they may continue to interact at parks, grocery stores, in sports, etc.

\begin{itemize}
    \item $\beta_{oo}$: Transmission rate among open-school students (school + community).
    \item $\beta_{cc}$: Transmission rate among closed-school students (community only).
    \item $\beta_{co}$: Transmission rate from closed-school to open-school students (community mixing).
    \item $\beta_{oc}$: Transmission rate from open-school to closed-school students (community mixing).
\end{itemize}

Something of note is that (I expect) $\beta_{oo} > \beta_{cc}$ because open school students would still be getting exposed
to other students, whereas the closed school students would be getting far less exposure in school. Also it's worth mentioning that $\beta_{co} \approx \beta{oc}$ 
as they are really measuring the same thing.

\subsection{Agent-based SEIR Model Update}

Keeping with the same idea as before, but applying it to individuals instead of entire `systems'. Each individual gets treated as an agent who has their own attributes:

\begin{itemize}
    \item \textbf{Category}: S, E, I, or R.
    \item \textbf{School}: Open, Closed.
    \item \textbf{Age group}: Student, Adult, Elderly.
    \item \textbf{Location}: Which school, household, neighborhood.
    \item \textbf{Contact Network}: List of other agents they interact with.
\end{itemize}

We would then construct a network of agents that all have this information stored with each agent. They also contain information on their 
contacts from each place they visit (household, school, and community).

This model would allow you to track individual-level outcomes and more accurately model the transmission of infection on an individual basis and with much 
greater information like who, (probability of) when, and from whom. You would also inadvertently create a network graph of who knows who and who spends time with who, creating 
heterogeneous contact patterns, and reveal clusterings (households, school cliques, community centers).

I remember that when COVID-19 was around, I was in my final year of High School. When schools eventually closed I spent a large amount of my time indoors and online (mainly based on preference). 
However, I had many friends that took the opportunity of `no more school' to spend time with friends outside of school at each others houses or outside in parks, increasing their social exposure. 
So, we could also reasonably see some behavioral changes where certain contacts are reduced (because interactions between agents don't happen at school anymore) and increased interactions 
with other agents (friends spending time together outside of school).


\begin{table}[H]
    \centering
    \begin{tabular}{cc}
        \toprule
        Compartmental Model & Agent-Based Model \\
        \midrule
        Quick insight into overall trends & Tracking of specific transmission chains \\
        Overall trend tracking & Targeted interventions \\
        \bottomrule
    \end{tabular}
\end{table}

\section{Epidemiological Models}

\begin{figure}[H]
    \centering
    \begin{tabularx}{\textwidth}{@{}XX@{}}
        \subsection{Individuals Can Recover?}
        \textbf{Models: SIR, SIS, SIRS, SEIR}
        
        Models that include \textbf{R} (Recovery) indicate that individuals can recover from infection, though not necessarily with permanent immunity.
        As for the \textbf{SIS} model, individuals do recover but they do not develop immunity to the infection.
        &
        \subsection{Monotonically Decreasing Susceptibility?}
        \textbf{Models: SI, SIR, SEIR}
        
        These models feature monotonically decreasing susceptibility because individuals do not return to the \textbf{S} category. Once removed from the susceptible population, they remain permanently infected or recovered.
        \\[2em]
        
        \subsection{Permanent Immunity?}
        \textbf{Models: SIR, SEIR}
        
        Models that end with Recovery indicate that once an individual recovers, they acquire permanent immunity and will not become infected again.
        &
        \subsection{Equilibrium?}
        \textbf{Models: SIS, SIRS}
        
        Equilibrium between \textbf{S} and \textbf{I} is only possible when individuals can return to the susceptible category. These models allow such transitions, enabling a stable endemic state.
        \\[2em]
        
        \subsection{Repeated Infections?}
        \textbf{Models: SIS, SIRS}
        
        Models that end with \textbf{S} (Susceptible) indicate that individuals will either become infected again or lose their immunity after recovery, returning to the susceptible state.
        &
        \subsection{Natural Decline of Infection?}
        \textbf{Models: SIR, SEIR, SI}
        
        These models feature natural infection decline. In SIR and SEIR, individuals recover and the infection eventually ends/dies. The SI model is debatable: while everyone becomes infected, one could argue that if everyone is infected, no one is infected.
        \\
    \end{tabularx}
\end{figure}

\section{Ebola and COVID-19 Paper Review}

\subsection{\textit{Epidemic of Ebola}}

The paper ``Modeling the Impact of Interventions on an Epidemic of Ebola in Sierra Leone and Liberia,'' by C.M. Rivers et al., covered the destructive public health crisis surrounding Ebola in Sierra Leone and Liberia. By October 2014, over 8,000 cases had been reported across parts of West Africa. In response to this quickly developing catastrophe, this paper sought to forecast the epidemic's trajectory and evaluate different interventions that could possibly catch the forming epidemic at a critical moment, to allow policymakers that desperately needed evidence-based guidance, options.

\subsubsection{Summary}

Using a modified SEIR compartmental model, Rivers et al.\ forecasted the progression of the pandemic to assess the potential impact of various public health interventions. Publishing the study on November 6, 2014, when the outbreak was still in its infancy but exponentially growing in cases and threat, the international community (including the U.S.) was searching for options. This modified model utilized two additional components that were based on where many more exposures and infections were happening: Hospitals and Funerals. This structure, the SEIHFR model, accounts for the more specific ways that Ebola would be transferred: community transmission, nosocomial transmission, and funeral-practice related transmission.

Using data through September 2014, the baseline projection predicted approximately $\sim$117,000 cumulative cases in Liberia and $\sim$30,000 cases in Sierra Leone by the end of 2014, with no signs of slowing. The model estimated $R_0$ as approximately 2.2 for Sierra Leone and 1.78 for Liberia. The researchers modeled several intervention strategies: enhanced contact tracing (80\%, 90\%, and 100\% of patients traced and hospitalized, or removed from the community), improved infection control (25\%, 50\%, and 75\% reduction in hospital transmission through better PPE usage), elimination of funeral transmission, and a hypothetical pharmaceutical intervention improving survival rates by 25\%, 50\%, and 75\%.

The key findings revealed that transmission-reducing interventions could substantially reduce cases but would not peak the epidemic early. Even the most aggressive scenarios (100\% contact tracing with 75\% reduction in hospital transmission) would not produce immediate epidemic decline. The authors concluded that interventions required sustained, long-term commitment and that policymakers needed to prepare for an extended response requiring international resources.

\subsubsection{Modifications and Novelties}

While basic SEIR models use homogeneous infectious compartments, Rivers et al.\ recognized that Ebola's unique transmission characteristics required a more nuanced structure. The Hospitalized (H) compartment acknowledges that patients receiving medical care have fundamentally different transmission dynamics than community cases. Hospitalized patients pose less risk to the general population but can expose healthcare workers if infection control is inadequate. And given that Liberia and Sierra Leone's healthcare system was in poorer condition than first world countries, the use of PPE was degraded or improper, leading to hieghtened transmission.

The Funeral (F) compartment represents the most culturally specific innovation. West African funeral practices involve direct contact with deceased bodies (washing and touching them as part of customs). Since Ebola remains highly infectious in bodily fluids after death, these practices created a significant transmission pathway. This compartment would be unnecessary for respiratory diseases but was essential for Ebola in this cultural context, demonstrating that effective epidemiological modeling must account for local practices, not just biological mechanisms, as discussed in the ``Discussion'' section.

Unlike standard SEIR models with a single transmission parameter ($\beta$), the SEIHFR model incorporates setting-specific rates: $\beta_C$ (community), $\beta_H$ (hospital), and $\beta_F$ (funeral). This separation allows the model to capture how interventions targeting specific pathways would affect overall epidemic dynamics. Each setting has different contact patterns and intervention opportunities, making this disaggregation crucial for policy guidance.

Rivers et al.\ faced the challenge of parameterizing their model using incomplete, rapidly changing data from an ongoing crisis, fitting the model to real-time WHO case reports while accounting for underreporting. Rather than simply projecting natural epidemic progression, they combined and parallelized intervention parameters (contact tracing 80--100\%, hospital transmission reductions 25--75\%), transforming the model into a decision-support framework. The paper also incorporated stochastic forecasting with 250 model iterations, presenting distributions of possible outcomes rather than point estimates, providing more realistic uncertainty quantification.

\subsubsection{Policy Implications and Applications}

The paper provided actionable guidance for policymakers during a critical period when intervention strategies were being designed and resources allocated. The model's most important policy contribution was demonstrating that transmission-reducing interventions, particularly enhanced contact tracing and improved hospital infection control, would have substantially greater impact than interventions primarily affecting case fatality rates. The hypothetical pharmaceutical intervention improving survival by 75\% had modest effects on total cases compared to interventions reducing transmission rates. This finding suggested that policymakers should prioritize rapid case identification and isolation through contact tracing, construction of Ebola Treatment Units, provision of adequate PPE for healthcare workers, and community education on safe burial practices. It also speaks to the saying ``An ounce of prevention is worth a pound of cure''---Benjamin Franklin. It significantly slows the spread to prevent it from getting out of hand in the first place.

Perhaps the paper's most heart-breaking contribution was limiting unrealistic expectations. Even aggressive scenarios did not produce immediate epidemic decline, with continued growth predicted for weeks or months after intervention upscale. This insight helped policymakers understand the need for sustained, long-term responses and provided justification for continued resource requests even as interventions were implemented. By modeling Sierra Leone and Liberia separately, the paper also demonstrated that optimal strategies might differ between countries, suggesting one-size-fits-all international responses might be suboptimal. 

\subsubsection{Critique and Limitations}

While Rivers et al.\ produced valuable work under challenging circumstances, several limitations popped out and are worth discussing. Case reporting during the outbreak was incomplete, with significant underreporting due to limited diagnostic capacity and remote geography. There was also a lack of reliable reports because the chaos of the outbreak made it infeasible to have consistent and accurate contact tracing and reports while also keeping professionals and community members safe. While the authors attempted to account for this through model fitting, baseline data inaccuracy necessarily affects all downstream predictions. More sophisticated approaches might incorporate explicit underreporting parameters varying by region and time, or use multiple data sources to cross-validate case counts.

Secondly, the compartmental model assumes homogeneous mixing, where any susceptible individual has equal probability of encountering any infectious individual. In reality, Ebola transmission was highly clustered geographically and within social networks. Outbreaks occurred in specific districts while other areas remained unaffected. Families caring for sick relatives experienced concentrated risk, while communities removed from epicenters in the countryside likely experienced a significantly lower risk of exposure. This spatial and social heterogeneity means contact patterns were far from random, and a compartmental model cannot capture these clustering effects, which have important implications for targeted interventions. The paper did account for this by having the ability to adjust transmission rates in the different compartments of the model, but this doesn't remove the underlying assumption of homogeneous mixing. 

The model assumes constant transmission parameters under each scenario. However, human behavior changes dramatically during epidemics. As awareness increased, communities likely modified behaviors, like reducing contact, avoiding funerals, self-isolating when symptomatic, and this would likely happen even without formal interventions. Conversely, some communities resisted interventions due to mistrust. These dynamic behavioral responses could substantially alter transmission rates in ways the model cannot capture. Incorporating behavioral feedback mechanisms would provide more realistic predictions. I remember during the COVID-19 pandemic that I was mainly keeping myself indoors and communicating with friends online instead of in person. In my social circle, their responses were split. Despite school closures and a mandate of sorts asking people to stay home, some viewed this as an opportunity to socialize even more. ``If I don't have to go to school for 7 hours, I can hang out with friends for 7 hours!'' was some of their mindsets, while others took it seriously. Point being, individuals won't necessarily heed official warnings in the ways that would be most beneficial to the greater ``herd.''

Finally, the paper evaluates interventions purely in epidemiological terms (cases averted) without considering implementation costs or cost-effectiveness. Policymakers must allocate limited budgets across competing interventions. Full decision analysis would include costs of contact tracing programs, PPE procurement, treatment unit construction, and safe burial teams that heed cultural customs, allowing formal cost-benefit analysis.

Rivers et al.'s work represents an important contribution to real-time epidemic modeling during a major public health crisis. The SEIHFR model's innovations---particularly explicit representation of hospital and funeral transmission pathways---demonstrate sophisticated understanding of disease-specific and context-specific factors. The systematic exploration of intervention scenarios provided valuable guidance to policymakers with limited resources.

\subsection{\textit{COVID-19 Mobility Modeling \& Policy Response}}

The paper ``Supporting COVID-19 policy response with large-scale mobility-based modeling'' by Chang et al. addresses the critical challenge of balancing public health protections with economic restrictions during the COVID-19 pandemic. Published in March 2021, this work seeks to create a practical decision-support tool for policymakers. The research was motivated by interactions with the Virginia Department of Health (VDH) and focuses on providing real-time analysis of how changes in mobility to specific venues affect COVID-19 transmission rates.

\subsubsection{Summary}

Chang et al.\ developed a decision-support dashboard that integrates large-scale mobility data with epidemiological modeling to quantify the relationship between location occupancy restrictions and COVID-19 infection rates. The system builds upon the SEIR (Susceptible-Exposed-Infectious-Removed) model by incorporating several critical enhancements and scaling it to support policy analysis across Virginia's three largest metropolitan statistical areas (MSAs): Washington-Arlington-Alexandria DC ($\sim$9.2M people), Virginia Beach-Norfolk-Newport News (``Eastern,'' $\sim$2.9M people), and Richmond ($\sim$2.0M people).

The model uses fine-grained mobility data to track movements from over 7,600 census block groups (CBGs) to 63,744 individual points of interest (POIs) such as restaurants, gyms, and retail stores, creating over 3 billion hourly edges in the mobility network. The modeling approach models infection risk at each POI depending on three factors: (1) transmission rate based on dwell time and venue area (size), (2) density of infectious visitors, and (3) mask-wearing rates. Critically, the model can simulate heterogeneous reopening scenarios where different POI categories (restaurants, essential retail, gyms, religious organizations, non-essential retail) operate at different capacity levels.

Key findings from their policy experiments include substantial variation in infection risk across POI categories. For example, returning Washington DC restaurants to 100\% of 2019 mobility levels would cause a 179\% increase in predicted infections with only a 34\% increase in visits, while returning essential retail to 100\% capacity would cause only an 8\% increase in infections with a 4\% increase in visits. The model also revealed that lower-income neighborhoods experienced 30--60\% higher infection rates than higher-income neighborhoods, attributable solely to differences in mobility patterns---lower-income residents reduced mobility less sharply and visited more crowded venues.

To make these insights actionable, the team built an infrastructure capable of running hundreds of thousands of model realizations in parallel (compressing 2 years of compute time into days) and designed a dashboard interface. The dashboard allows policymakers to adjust mobility levels for five POI categories to different capacities, immediately visualizing the predicted impact on infections and economic activity over 1--4 week periods across Virginia's regions.

\subsubsection{Modifications and Novelties}

Unlike traditional compartmental models that use aggregate measures of mobility to modulate transmission rates uniformly, Chang et al.\ explicitly model individual movements at with high granularity. The mobility network captures not just that people reduced travel, but specifically \textit{where} they went, \textit{when} they went, and \textit{how long} they stayed. This granularity enables the model to differentiate infection risk between a crowded small restaurant and a spacious grocery store, even if both see similar visit numbers. The transmission rate at each POI is calculated as $\psi_p^{(t)} = \kappa d_p^2 (v_p^{(t)}/a_p)$, where $d_p$ is median dwell time, $v_p^{(t)}$ is visitor count, $a_p$ is physical area, and $\kappa$ is a fitted transmission constant. This formulation captures that longer stays in more crowded, smaller spaces substantially increase transmission risk.

The paper also introduces four critical time-varying elements: (1) \textit{mask-wearing variation} scaling transmission to the proportion of people wearing masks, (2) \textit{dynamic base transmission rate} $\beta_{\text{base}}(t)$ capturing infections outside POIs (households, unmeasured venues), (3) \textit{time-varying death detection rate} $p_{\text{detect}}(t)$ estimated from NCHS excess death data, increasing from 20\% early in the pandemic to 60--80\% by late 2020; and (4) \textit{time-varying infection fatality rate} $p_{\text{IFR}}(t)$ reflecting improved treatments that halved mortality by summer 2020. These additions allow the model to accurately fit death curves despite dramatic behavioral changes over time.

Next, rather than randomly initializing disease states, the model uses historical reported deaths to estimate the proportion of each CBG's population in susceptible, exposed, infectious, and removed states at the start. This methodology back-calculates from deaths (accounting for the 18-day lag from infection to death reporting) to infer infection histories, providing more realistic starting conditions especially for second-wave modeling where cumulative infection rates varied substantially across neighborhoods.

Something I found incredibly impressive was that the team developed systems to support policy analysis at scale. Their vectorized model implementation parallelizes computations across POIs, CBGs, and stochastic realizations, reducing runtime from months or even years down to days or even less. More impressively, they built infrastructure to run hundreds of models in parallel. This computational capability transforms the model from a research tool into an operational decision-support system.

And in addition to the impressive computational model system they created, they made a dashboard that represents a significant advance in translating the complex model into actionable policy guidance. Unlike static reports, the interface provides immediate feedback on thousands of potential policies through integrated panels: (1) a POI Navigation Bar with sliders for setting target mobility levels, showing current mobility heterogeneity via median and interquartile ranges; (2) a Map Panel color-coding MSAs by infection impact with hover-over details; (3) a Chart Panel displaying time series of predicted infections under current versus target mobility with 95\% confidence intervals; (4) a Data Table quantifying differences across regions; and (5) a Mobility History Panel contextualizing current mobility relative to 2019 baselines. This design makes sophisticated modeling accessible to policymakers without technical expertise.

\subsubsection{Policy Implications and Applications}

The model shows that not all businesses pose equivalent transmission risk. Restaurants were particularly high-risk, meaning that returning them to full capacity would increase infections far more than visits gained, whereas essential retail showed much more favorable trade-offs. This finding supports differentiated reopening strategies rather than blanket restrictions. The model's ability to test specific combinations (restaurants at 50\%, gyms closed, essential retail at 100\%) enables optimization of this balance.

The research also demonstrates that occupancy restrictions can be more effective than proportional mobility reductions across all locations. A capacity limit at high-risk POIs produces greater infection reduction than reduction in visits across all POIs, because transmission scales non-linearly with density. This insight justifies policies that cap venue occupancy rather than simply reducing overall mobility, providing a more targeted and potentially less economically damaging intervention strategy.

By revealing that lower-income neighborhoods' higher infection rates stem from mobility patterns---less ability to reduce visits and patronage of more crowded venues---the model illuminates mechanisms of health inequity. This finding suggests interventions should target support for disadvantaged communities (e.g., paid sick leave, delivery services, occupancy limits at POIs in lower-income areas) rather than assuming uniform behavioral responses to general restrictions.

The dashboard's 1--4 week forecast horizons align with typical policy implementation timelines, making predictions actionable for near-term decision-making. The ability to immediately test ``what-if'' scenarios enables iterative policy refinement. The visualization of uncertaintys appropriately communicates prediction reliability, crucial for risk-aware decision-making.

The model's demonstration that even aggressive interventions don't immediately reverse epidemic growth (like in the previous paper) provides critical justification for sustained political and financial commitment.

\subsubsection{Critique and Limitations}

SafeGraph mobility data, derived from smartphone applications, systematically under-represents certain populations, like children (who may not carry phones), elderly individuals (lower smartphone adoption), and low-income individuals (less likely to have smartphones or location-enabled apps). The authors acknowledge this but cannot fully quantify the resulting biases. More critically, SafeGraph captures only specific POI visits, missing transmission in private homes, outdoor gatherings, workplaces without storefronts, and many other venues. The ``base transmission rate'' attempts to capture these unmeasured pathways, but lumping diverse transmission mechanisms into a single parameter likely oversimplifies complex dynamics.

The model also makes some serious but necessary assumptions about transmission. Assuming that all visitors at POI's at a given hour face identical infection risk ignores any amount of clustering, ventilation, air circulation, and indoor/outdoor settings. These factors were recognized and pushes towards the middle of the pandemic as very important factors for transmission. 

The model does not incorporate vaccination, which was just beginning during the modeled November 2020--January 2021 period but became widespread by Spring 2021. Vaccines create heterogeneity in susceptibility not captured by the homogeneous mixing assumption within CBGs. While these factors didn't affect the time period studied, they limit the model's utility for forward-looking policy in periods with significant vaccination changes.

Like the Ebola paper, this compartmental model assumes homogeneous mixing within CBGs, ignoring clustering in social and spatial networks. COVID-19 transmission occurs heavily within households, workplaces, and social circles. This model does a better job at working for granularity in this area, but the underlying structure of the model still expects homogeneous mixing.

\textbf{Suggestions for Improvement}

Several extensions could enhance the work: (1) \textit{Integration of additional data sources} including school attendance, workplace mobility, public transit usage, and household composition data to better capture unmeasured transmission pathways; (2) \textit{Explicit modeling of ventilation and indoor/outdoor distinctions}, potentially using venue-level attributes or seasonal adjustments; (3) \textit{Behavioral feedback mechanisms} where mobility responds to local infection rates, creating more realistic dynamics; (4) \textit{Validation against intervention natural experiments} comparing predictions to actual outcomes when specific jurisdictions implemented modeled policies; (5) \textit{Extension to vaccination periods} incorporating heterogeneous immunity and breakthrough infections.

\end{document}